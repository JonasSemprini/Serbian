\documentclass[12pt,
               a4paper,
               article,
               oneside,
               english,oldfontcommands]{memoir}
\usepackage{student}

% Metadata
\date{\today}
\setmodule{Colloquial Serbian}
\setterm{Vår, 2023}

%-------------------------------%
% Other details
% TODO: Fill these
%-------------------------------%
\title{Serbian Practice Notes}
\setmembername{Jonas Semprini Næss}  % Fill group member names

%-------------------------------%
% Add / Delete commands and packages
% TODO: Add / Delete here as you need
%-------------------------------%
\makeatletter
\newcommand*{\rom}[1]{\expandafter\@slowromancap\romannumeral #1@}
\makeatother
\usepackage{setspace}
\usepackage[T1]{fontenc}
\usepackage{titling}% the wheel somebody else kindly made for us earlier
\usepackage{fancyhdr}
\usepackage{fancybox}
\usepackage{epigraph} 
\usepackage{tikz}
\usepackage{pgfplots}
\pgfplotsset{compat=1.12}
\usepackage{lmodern}
\usepackage{enumitem}
\usepackage{caption}
\usepackage{subcaption}
\usepackage{fancyvrb}
\usepackage[scaled]{beramono}
\usepackage[final]{microtype}
\usepackage{amssymb}
\usepackage{multirow}
\usepackage{mathtools}
\usepackage{amsthm}
\usepackage{thmtools}
\usepackage{babel}
\usepackage{csquotes}
\usepackage{listings}
\usetikzlibrary{calc,intersections,through,backgrounds}
\usepackage{tkz-euclide} 
\lstset{basicstyle = \ttfamily}
\usepackage{float}
\usepackage{textcomp}
\usepackage{siunitx}
\usepackage{xcolor}
\usepackage{graphicx}
\usepackage[colorlinks, allcolors = uiolink]{hyperref}
\usepackage[noabbrev]{cleveref}
\pretolerance = 2000
\tolerance    = 6000
\hbadness     = 6000
\newcounter{probnum}[section]
\newcounter{subprobnum}[probnum] 
\usepackage{dirtytalk}
\usepackage{listings}
\usepackage{xcolor}
\usepackage{caption}
\usepackage[section]{placeins}
\usepackage{varwidth}
\usepackage{optidef}
\definecolor{uiolink}{HTML}{0B5A9D}
\definecolor{dkgreen}{rgb}{0,0.6,0}
\definecolor{gray}{rgb}{0.5,0.5,0.5}
\definecolor{mauve}{rgb}{0.58,0,0.82}
\lstset{frame=tb,
  language=R,
  aboveskip=3mm,
  belowskip=3mm,
  showstringspaces=false,
  columns=fullflexible,
  basicstyle={\small\ttfamily},
  numbers=none,
  numberstyle=\tiny\color{gray},
  keywordstyle=\color{blue},
  commentstyle=\color{dkgreen},
  stringstyle=\color{mauve},
  breaklines=true,
  breakatwhitespace=true,
  tabsize=3
} 
\usepackage{commath}
\newtheorem{theorem}{Theorem}[section]
\newtheorem{corollary}{Corollary}[theorem]
\newcommand{\Q}{ \qquad \hfill \blacksquare}
\newcommand\myeq{\stackrel{\mathclap{\normalfont{uif}}}{\sim}}
\let\oldref\ref
\renewcommand{\ref}[1]{(\oldref{#1})}
\newtheorem{lemma}[theorem]{Lemma}
\setlength \epigraphwidth {\linewidth}
\setlength \epigraphrule {0pt}
\AtBeginDocument{\renewcommand {\epigraphflush}{center}}
\renewcommand {\sourceflush} {center}
\parindent 0ex
\renewcommand{\thesection}{\roman{section}} 
\renewcommand{\thesubsection}{\thesection.\roman{subsection}}
\newcommand{\KL}{\mathrm{KL}}
\newcommand{\R}{\mathbb{R}}
\newcommand{\E}{\mathbb{E}}
\newcommand{\T}{\top}
\newcommand{\bl}{\left\{}
\newcommand{\br}{\right\}}
\newcommand{\spaze}{\vspace{4mm}\\}
\newcommand{\N}{\mathbb{N}}
\newcommand{\Rel}{\mathbb{R}}
\newcommand{\expdist}[2]{%
        \normalfont{\textsc{Exp}}(#1, #2)%
    }
\newcommand{\expparam}{\bm \lambda}
\newcommand{\Expparam}{\bm \Lambda}
\newcommand{\natparam}{\bm \eta}
\newcommand{\Natparam}{\bm H}
\newcommand{\sufstat}{\bm u}

% Main document
\begin{document}
\header{}
\section*{Upoznavanje (Getting to know)}
\emph{Dialogue 1:}\spaze
\textbf{(A) Dzon i Andjela Braun putuju u Beograd (Jon and Andjela traveling/going to Beograd)} \spaze
Milan: Dobar dan (Good day)\\[3pt]
Andjela: Dobar dan (Good day) \\[3pt]
Milan: Ja sam Milan Jovanovic(ch) (I am Milan Jovanovic) \\[3pt]
Andjela: Ja sam Andjela Braun, a ovo je moj muz (zh), Dzon. (I am Andjela Braun, and this is my husband, Jon) \\ [3pt]
Milan: Drago mi je. Dobro dosli u Beograd (Nice to meet yo. Welcome to Beograd) \spaze 
\textbf{(B)} \spaze
Dzon: Vi ste Srbin, zar ne? (You are serbian, aren`t you?) \\[3pt]
Milan: Da, ja sam Srbin, a vi? (Yes, I am serbian, and you?) \\[3pt]
Dzon: Mi smo Englezi. (We are english.) \spaze 
\textbf{Personal Pronouns:}
\begin{table}[H]
    \begin{center}
      \caption{Personal Pronouns}
      \begin{tabular}{l|S|r}
        \toprule % <-- Toprule here
        \textbf{Singular} & \textbf{Plural} & \textbf{English}\\
        \midrule % <-- Midrule here
        1st \ Ja  & Mi & I, \ We\\[4pt]
        2nd \ Ti & Vi & You(familiar), \ You(formal, polite)\\[4pt]
        3rd \ On & Oni & He, \ They(m. or mixed gender)\\[4pt]
        \ Ona & {One} & They(f.)\\[4pt]
        \ Ono & {Ona} & They(n.)\\[4pt]
        \bottomrule % <-- Bottomrule here
      \end{tabular}
    \end{center}
  \end{table}
  Personal pronouns are not used when they are the subject (nominative case), except for emphasis.\spaze 
  \textbf{Examples:}\spaze 
  Kako se zovete? (No personal pronoun) \\[3pt]
  Zovem se Milan. Kako se vi zovete? (I`m called Milan. What are you called?)\\[3pt]
  Ne razumeju engleski. Da li on govori srpski? (They don`t understand English. Does he speak Serbian?) \spaze 
  \emph{Dialogue 2:} \spaze 
  - Zdravo Andjela, kako si (Hello Andjela, how are you?) \spaze 
  - Dobro hvala, a kako si ti (Good thank you, and how are you?) \spaze 
  - I ja sam dobro, hvala (I am fine too, thank you) \spaze  
  \spaze 
  - Govorite li srpski? (Do you speak serbian?) \spaze 
  - Naz(zh)alost, jos(sh) ne. Ali uc(tch)imo (ucim)! (Unfortunately, not yet. But we`re learning) \spaze 
  - Bravo! \spaze 
  \subsection*{Biti (I am)}
  \begin{table}[H]
    \begin{center}
      \caption{Short form of Biti}
      \begin{tabular}{l|S|r}
        \toprule % <-- Toprule here
        \textbf{English} & \textbf{Serbian} & \textbf{to be}\\
        \midrule % <-- Midrule here
        \ I  & Ja & sam\\[4pt]
        \ You & Ti & si\\[4pt]
        \ He/she/it & On/ona/ono & je\\[4pt]
        \ We & Mi  & smo \\[4pt]
        \ You & Vi  & ste \\[4pt]
        \ They & {Oni/one/ona}  & su \\[4pt]
        \bottomrule % <-- Bottomrule here
      \end{tabular}
    \end{center}
  \end{table}
\subsection*{Exercises (Vez(zh)be)} 
(a) \spaze 
1.) Ja sam Engleskinja. Moj Muz (zh) je S(sh)kotlandjanin. (I am English (female). My husband is Scottish) \spaze 
2.) Oni su Amerikanci, a mi smo Englezi. (They are american, and we are English) \spaze 
3.) Vesna i Neda su Hrvatice. One su studentkinje (Vesna and Neda are croatians. They are students) \spaze 
4.) Vi ste Bosanac? (Are you (formal) Bosnian?) \spaze 
5.) Oni su dobro. (They are good.) \spaze 
(b) \spaze 
1.) Mi ne govorimo srpski (We don`t speak serbian) \spaze 
2.) Ti si Francuz. (You are french (male)) \spaze 
3.) Vi ste studenti? (Are you students?) \spaze 
4.) Andjela i Dzon su muz i zena. Oni su Englezi (Andjela and John are husband and wife. They are english.) \spaze 
5.) Zovem se Milan. Ja sam iz Beograda (My name is Milan. I am from Beograd) \spaze 
\subsection*{Biti (I am) long form}
  \begin{table}[H]
    \begin{center}
      \caption{Long form of Biti}
      \begin{tabular}{l|S|r}
        \toprule % <-- Toprule here
        \textbf{English} & \textbf{Serbian} & \textbf{to be}\\
        \midrule % <-- Midrule here
        \ I  & Ja & Jesam\\[4pt]
        \ You & Ti & Jesi \\[4pt]
        \ He/she/it & On/ona/ono & Jest(e)\\[4pt]
        \ We & Mi  & jesmo \\[4pt]
        \ You & Vi  & jeste \\[4pt]
        \ They & {Oni/one/ona}  & jesu \\[4pt]
        \bottomrule % <-- Bottomrule here
      \end{tabular}
    \end{center}
  \end{table}
  \subsection*{Exercise 2:} 
(a)\spaze 
1.) Je li Neda Srpkinja? - Jeste (Is Neda Serbian?)\spaze 
2.) Jesu li Dado i Denis Bosanci? - Jesu (Are Dado and Denis Bosnian?) \spaze 
3.) Jesmo li mi Srbi? -  Jesmo (Are we Serbs?) \spaze 
4.) Je li moj muz Irac? - Jeste (Is my husband Iranian?) \spaze 
5.) Jesi li ti Amerikanac - Jesam (Are you American?) 
\subsection*{Biti (I am) negative form}
  \begin{table}[H]
    \begin{center}
      \caption{Long form of Biti}
      \begin{tabular}{l|S|r}
        \toprule % <-- Toprule here
        \textbf{English} & \textbf{Serbian} & \textbf{to be}\\
        \midrule % <-- Midrule here
        \ I  & Ja & Nisam\\[4pt]
        \ You & Ti & Nisi \\[4pt]
        \ He/she/it & On/ona/ono & Nije\\[4pt]
        \ We & Mi  & Nismo \\[4pt]
        \ You & Vi  & Niste \\[4pt]
        \ They & {Oni/one/ona}  & Nisu \\[4pt]
        \bottomrule % <-- Bottomrule here
      \end{tabular}
    \end{center}
  \end{table}
  \subsection*{Exercise 3:}
  1.) Mi nismo studenti - (We are not students) \spaze 
  2.) Oni nisu muz i zena - (They are not husband and wife) \spaze 
  3.) Ti nisi Englez - (You are not English) \spaze 
  4.) On nije moj muz - (He is not my husband) \spaze 
  5.) Ja nisam dobro - (I am not good) \spaze  
  \subsection*{Exercise 4:} 
  - Zdravo, ja sam Marko. A kako se ti zoves? (Hello, I am Marko. And what is your name?) \spaze 
  - Zdravo, ja se zovem Dzon. (Hello, my name is Dzon) \spaze 
  - Drago mi je. (Nice to meet you) \spaze 
  - Drago mi je. \spaze 
  - Jesi li ti Englez, Dzon? (Are you English, John?) \spaze 
  - Nisam. Ja sam Amerikanac. A ti Marko? (I am not. I am American. What about you Marko) \spaze 
  - Ja sam Srbin. (I am Serbian.) 
  \subsection*{Exercise 5:}
  - Zdravo Vesna (Hello Vesna) \spaze 
  - Cao, Dzon. Cao Andjela (Hello, John and Andjela) \spaze 
  - Kako ste? (How are you?) \spaze
  - Mi smo dobro. A ti? (We`re good, and you?) \spaze
  - Nisam lose (Not bad) \spaze
  \subsection*{Formation of questions:} 
  \textbf{\textit{With interrogative words}:} \spaze 
  Zasto ucite srpski? (*Why* are you learning serbian?) \spaze 
  Ko ste vi? (*Who* are you?) \spaze 
  Sta radis ovde? (*What* are you doing here?)\spaze 
   \textbf{\textit{No interrogative word (verb then the particle *li*}:} \spaze 
   Dolazite li cesto ovamo (*Do you* come here often?) \spaze 
   Ceka li vasa zena (Is your wife waiting?) \spaze 
    \textbf{\textit{No interrogative word with the conjunction da.(the particle *li* and verb can be placed anywhere}:}\spaze 
    Da li govorite srpski? (Do you speak serbian?) \spaze 
    Da li razumete engleski? (Do you understand english?) \spaze 
     \textbf{\textit{Negative questions (Zar)}:}\spaze
     Zar ti nisi Englez? (Aren`t you English?) \spaze 
     Zar studenti ne razumeju? (Don`t the students understand?)\spaze 
  \subsection*{Exercise 6:} 
  \textbf{a.)} \spaze
 	1.) Da li Ben iz Londona? - Da, jeste (Is Ben from London? Yes, he is) \spaze 
 	2.) Da li je on sada u Parizu? Ne, on je u Beogradu. (Is he in Paris now? No, he is in Beograd.) \spaze 
 	3.) Ko ceka Bena na aerodromu? Dejan na aerodromu ceka Bena. (Who is waiting for Ben at the airport? Dejan is waiting for Ben at the airport.) \spaze 
 	4.) Gde odlaze Ben i Dejan? - Oni su odlaze u Zemun (odlaze u Zemun) (Where are Ben and Dejan leaving? They are leaving for Zemun.) \spaze 
 \textbf{b.)} \spaze 
Carinik - Dobar dan, gospodine Vilson. (Good day, mister Vilson). \spaze
Ben - Dobar dan. (Good day) \spaze 
Carinik - Vas pasos, molim. (Your passport, please.) \spaze 
Ben - Izvolite (Here you go) \spaze 
Carinik - Sve je u redu. Dobro dosli u Beograd (Everything is okay. Welcome to Beograd) \spaze 
Ben - Hvala. (Thank you)
\end{document}